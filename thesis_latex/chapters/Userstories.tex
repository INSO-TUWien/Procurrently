\section{User stories}
This section describes user centered requirements in the form of user stories.
\begin{enumerate}
\item As a programmer, I want to access the entire project structure when using a code editor.
\item As a programmer, I want a clean separation between branches and do not want to see changes to other branches when using a version control system.
\item As a programmer, I want to be able to stage changes that I made to Git.
\item As a project manager, I want to see who made a specific modification to a project in the version control system.
\item As a programmer, I do not want to configure a second version control system when I already provided this information to Git.
\item As a programmer, I do not want files covered by my .gitignore configuration to be shared with others.
\item As a programmer, I want to be able to edit files with up to 4 people at a time. 
\item The system processes files up to 30.000 characters.
\end{enumerate}

\section{Scenarios}
\label{sec:scenarios}
In this section, multiple scenarios are derived from the user stories.

\subsection{Scenario 1}
\label{sec:scenario1}
Person A and B are working on a Node.js Express project together. Person A is working on branch "Ba", and person B is working on Branch "Bb".
Person A has edited the file "/app.js" (paths are described as absolute from the project root) as well as the file "/routes/account.js" on "Ba".
Person B has edited the file "/app.js" as well as the file "/routes/main.js" on "Bb".
Person A encounters unexpected behaviour of his code. He requests help from person B.
Neither persons A nor B want to commit their changes at this point.
Person B uses the VS Code Command Palette and chooses the option "Procurrently: Checkout Branch" and chooses branch "Ba".
Person B now sees only the modifications in "/app.js" and /routes/accounts.js from branch "Ba".
Upon discovering the problem in the source code, person B modifies "/routes/account.js" to solve the problem.
Person A can see the changes in real-time.
Person B once again opens the VS Code Command Palette and chooses the option "Procurrently: Checkout Branch" and chooses branch "Bb".
Person B now sees the modifications in "/app.js" and "/routes/main.js" on Branch "Bb" and can continue working on Person B's task.

\subsection{Scenario 2}
Person A and B are working on an Node.js Express project together. Both of them are working on branch "master" in the "/app.js" file.
Person A is modifying and testing a new feature while person B is working on setting new HTTP caching headers for the Express app.
This is interfering with the test of person A. Person A opens the VS Code Command Palette and chooses the option "Procurrently: Toggle remote changes".
All the modifications of B are reverted in the documents of person A.
During that person B can still see the modifications of person A which helps person B to determine the correct caching headers for the different routes.
Person A's laptop disconnects from the network. Therefore, person B can no longer receive real-time updates from person A.
Person A's laptop reboots unexpectedly. Once the laptop is booted up again, person A opens VS Code and can continue to work where person A left off.
Person A now wants to know about the progress of person B and turns on remote changes. Person A can see all the changes of person B until person A disconnected from the network. Person A disables remote changes again and continues to work on his new feature.
Later on, person B's laptop reconnects to the network and person B can receive all the modifications person B missed.
Once person A is happy with the implementation of the new feature, person A wants to test the caching behaviour himself to make sure person B has understood his code correctly.
Person A opens the VS Code Command Palette and chooses the option "Procurrently: Toggle remote changes".
Person A now sees all the modifications of person B again including the modifications B made while person A had disabled the remote changes.
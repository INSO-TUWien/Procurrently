%%%%%%%%%%%%%%%%%%%%%%%%%%%%%%%%%%%%%%%%%%%%%%%%%%%%%%%%%%%%%%%%%%%%%%%%
\chapter{Introduction}
\label{sec:introduction}
%%%%%%%%%%%%%%%%%%%%%%%%%%%%%%%%%%%%%%%%%%%%%%%%%%%%%%%%%%%%%%%%%%%%%%%%

%=======================================================================
\section{Problem Description}
%=======================================================================

When multiple people are working on solving a Problem it is not clear who is implementing which part. Current vcs systems conflict resolutions require manual conflict resolution if multiple people have modified the same file.
In order for a group to discuss a problem all the code has to be commited and pulled by everyone first. This introduces friction. In order to solve these and other problems real time collaborative editors have been created.
But current implementations of real time collaboration editors / editor plugins are unaware of the underlying version control system and therefore are based on the idea of just sharing files of a host mashine or a single source of truth file on a server instead of basing edit histories on versions of files known to the version control system anyway.

In state of the art collaborative code editors it is not possible for a user to easily see who made specific changes. Usually all the changes are bundled in one commit and the accountability is lost. In order to convert this concurrent model to a commit understood by Git it should be possible to stage changes by author. This approach might require more sophisticated logic than just interpreting changes as strings given that a command could be modified by two users and applying only half the changes as part of a commit could result in invalid syntax or semantic.

\section{Expected Results}

Enable real time collaboration on source code based on a Git\footnote{\href{https://git-scm.com/}{https://git-scm.com/}} project by
continuous tracking of code changes synchronizing over peer to peer connection.
Lowering overhead of splitting tasks by enabling everyone see what other people are working on. Allowing discussions about source code that has not yet been commited.
Changes should be commitable by author. In order to be able to compile sourcecode, changes of other people can be toggled off.
Using Git as a base enables opportunistic real time collaboration. In other words if a connection is possible changes will be propagated  to other people working on the same branch. If not the changes will be sent when a connection is available. \cite{AlwisSillito:2009:centralToDecentralVCS} Therefore it should be analog to the benefits of moving to a decentralized vcs.

%=======================================================================
\section{Motivation}
%=======================================================================

Current implementations of real time collaboration tools are not designed with version control systems in mind.

Transforming real time collaborative edits into regular Git commits by author will reduce a lot of friction in the adoption for real time collaboration software. Using a peer to peer solution with the ability to deal with disconnect events using information already known to the version control system will drasticly increase the ease of use. Ideally a user won't even have to think about using the extension.

In diesem Kapitel wird der Forschungsbedarf aufgezeigt. Nach dem Lesen dieses Kapitels sollten folgende Punkte klar dargestellt sein:
\begin{itemize}
	\item Aktueller Stand der Wissenschaft in Bezug auf die zuvor formulierte Problemstellung und klare Darstellung, was hier unzureichend/offen ist.
	\item GGf. Darstellung des größeren Forschungsbereichs, in den die Diplomarbeit eingebettet ist.
	\item Darlegung der Bedeutung des Themas für den Stand oder die Weiterentwicklung eines Bereichs der Informatik (z.B. Datenbanksysteme, Mobile Anwendungen, Java-Programmierung, Rechenzentrumsbetrieb, \dots) oder eines Fachbereichs (z.B. Bankwesen, Wertpapierhandel, Gesundheitswesen, Transportwesen, Flugsicherheit \dots). Erklärung, was durch die Lösung des Problems z.B. kostengünstiger/schneller/hochwertiger/sicherer/anwendbarer/\enquote{schöner} etc. wird.
\end{itemize}

%=======================================================================
\section{Zielsetzung}
%=======================================================================

Nachdem die Problemstellung und die Motivation zu deren Lösung formuliert wurden, wird in diesem Kapitel das zu erarbeitende Resultat beschrieben.

\makeatletter\ifthesis@masterthesis
Nach dem Lesen dieses Kapitels sollten folgende Punkte klar dargestellt sein:
\begin{itemize}
	\item Umfang, in dem die Problemstellung am Ende der Arbeit gelöst sein soll bzw. mit welchen Einschränkungen.
	\item Methode zur Erarbeitung des Resultats.
	\item Gibt es einen Theorie- und einen Praxisteil?
	\item Schwerpunkte des Praxisteils (z.B. Durchführung einer Umfrage, Programmierung, Herstellung von Hardware, Erprobung einer Methode in einem konkreten Projekt)?
	\item Art des Resultats (z.B. ein Programm, eine Formel, eine Methode, die Erweiterung einer existierenden Methode, ein Konzept, ein Framework, Hardware-Prototyp, eine bewiesene Erkenntnis)?
\end{itemize}
\fi\makeatother

\makeatletter\ifthesis@masterthesis
%=======================================================================
\section{Aufbau der Arbeit}
%=======================================================================

Beispielhaft:

Kapitel \ref{sec:fundamentals} behandelt sowohl Grundlagen als auch Definitionen und bietet einen Überblick \dots, die als Basis für \dots dienen.

\dots, wird in Kapitel 3 erläutert..

Ein Anwendungsszenario (Fallbeispiel), das \dots, ist in Kapitel 4 dargestellt. Dieses Szenario umfasst \dots.

Kapitel 5 setzt sich \dots. auseinander.

Einsatzmöglichkeiten in der Praxis werden in Kapitel 6 diskutiert.

Abschließend fasst Kapitel \ref{sec:conclusion} die wesentlichen Erkenntnisse zusammen und gibt einen Ausblick in die Zukunft.
\fi\makeatother

%%%%%%%%%%%%%%%%%%%%%%%%%%%%%%%%%%%%%%%%%%%%%%%%%%%%%%%%%%%%%%%%%%%%%%%%%%%%%%%%%%%%%%%%%%%%%%%%%%%%%%%%%%%%%%%%%%%%%%%%%%%%%%%%%%%%%%%%%%%%%%%%%%%%%%%%%%%%%%%%%%%%%%%%%%
% \section{General Information}
%
% This document is intended as a template and guideline and should support the author in the course of doing the master's thesis.
% Assessment criteria comprise the quality of the theoretical and/or practical work as well as structure, content and wording of the written master's thesis. Careful attention should be given to the basics of scientific work (e.g., correct citation).\footnote{Sample Footnote}
%
% \section{Organizational Issues}
%
% A master's thesis at the Faculty of Informatics has to be finished within six months. During this period regular meetings between the advisor(s) and the author have to take place.
% In addition, the following milestones have to be fulfilled:
% \begin{enumerate}
%   \item  Within one month after having fixed the topic of the thesis the master's thesis proposal has to be prepared and must be accepted by the advisor(s). The master's thesis proposal must follow the respective template of the dean of academic affairs. Thereafter the proposal has to be applied for at the deanery. The necessary forms may be found on the web site of the Faculty of Informatics. \url{http://www.informatik.tuwien.ac.at/dekanat/formulare.html}
%   \item  Accompanied with the master's thesis proposal, the structure of the thesis in terms of a table of contents has to be provided.
%   \item Then, the first talk has to be given at the so-called ``Seminar for Master Students''. The slides have to be discussed with the advisor(s) one week in advance. Attendance of the ``Seminar for Master Students'' is compulsory and offers the opportunity to discuss arising problems among other master students.
%   \item At the latest five months after the beginning, a provisional final version of the thesis has to be handed over to the advisor(s).
%   \item As soon as the provisional final version exists, a first poster draft has to be made. The making of a poster is a compulsory part of the ``Seminar for Master Students'' for all master studies at the Faculty of Informatics. Drafts and design guidelines can be found at \url{http://www.informatik.tuwien.ac.at/studium/richtlinien}.
%   \item After having consulted the advisor(s) the second talk has to be held at the ``Seminar for Master Students''.
%   \item At the latest six months after the beginning, the corrected version of the master's thesis and the poster have to be handed over to the advisor(s).
%   \item After completion the master's thesis has to be presented at the ``epilog''. For detailed information on the epilog see: \\ \url{http://www.informatik.tuwien.ac.at/studium/epilog}
% \end{enumerate}
%
% \section{Structure of the Master's Thesis}
%
% If the curriculum regulates the language of the master's thesis to be English (like for ``Business Informatics''), the thesis has to be written in English. Otherwise, the master's thesis may be written in English or in German. The structure of the thesis is predetermined.
% The table of contents is followed by the introduction and the main part, which can vary according to the content. The master's thesis ends with the bibliography (compulsory) and the appendix (optional).
%
% \begin{itemize}
%   \item	Cover page
%   \item Acknowledgements
%   \item Abstract of the thesis in English and German
%   \item Table of contents
%   \item Introduction
%   	\begin{itemize}
%   		\item motivation
%   		\item problem statement (which problem should be solved?)
%   		\item aim of the work
%   		\item methodological approach
%   		\item structure of the work
%   	\end{itemize}
%   \item State of the art / analysis of existing approaches
%   	\begin{itemize}
%   		\item literature studies
%   		\item analysis
%   		\item comparison and summary of existing approaches
%   	\end{itemize}
%   \item Methodology
%   	\begin{itemize}
%   		\item used concepts
%   		\item methods and/or models
%   		\item languages
%   		\item design methods
%   		\item data models
%   		\item analysis methods
%   		\item formalisms
%   	\end{itemize}
%   \item Suggested solution/implementation
%   \item Critical reflection
%   	\begin{itemize}
%   		\item comparison with related work
%   		\item discussion of open issues
%   	\end{itemize}
%   \item Summary and future work
%   \item Appendix: source code, data models, \dots
%   \item Bibliography
% \end{itemize}
%

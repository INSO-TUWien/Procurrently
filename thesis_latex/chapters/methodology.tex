\chapter{Methodological Approach}

First, a state-of-the-art analysis was conducted. This analysis was looking into currently available open-source and commercial products for real-time collaboration. 
Additionally, the current literature related to real-time collaboration was studied.
The papers mentioned in open source projects were used as a starting point for forward and backward literature analysis.


The following list of publishers were used as sources of literature: ACM Digital Library\footnote{\href{https://dl.acm.org/}{https://dl.acm.org/}}, IEEE Xplore Digital Library\footnote{\href{https://ieeexplore.ieee.org/}{https://ieeexplore.ieee.org/}}, SpringerLink\footnote{\href{https://link.springer.com/}{https://link.springer.com/}}, and ScienceDirect\footnote{\href{https://www.sciencedirect.com/}{https://www.sciencedirect.com/}}. Google Scholar\footnote{\href{https://scholar.google.com/}{https://scholar.google.com/}} was used as an additional search engine. Papers that are not published by the previously mentioned publishers were only used in exceptional cases and after consultation with the supervising assistant.


The literature survey was continuously documented in the following format:

\begin{lstlisting}
	<date>: <search query>
		<publisher>
			<paper title>
	<search query> := one of the following:
\end{lstlisting}

\begin{itemize}
    \item search terms
    \item conference
    \item forward/backward search paper title
    \item authors of specific papers 
\end{itemize}
If multiple searches occurred on the same date, the date has to be included again for every search query.

Based on problems mentioned in papers and issues of projects from the state-of-the-art analysis, requirements for the extension were defined.
These requirements are user-centred.
Functional as well as non-functional aspects such as performance were defined as requirements.

Use cases for the extension were also defined. These use cases directly related to editing source code have a defined number of users as well as file sizes to be verifiable. 
Every use case includes at least the following information:
\begin{itemize}
    \item Actor (the stakeholder)
    \item System (the software the actor is interacting with)
    \item Action (the goal of the actor)
\end{itemize}
A VS Code\footnote{\href{https://code.visualstudio.com/}{https://code.visualstudio.com/}} extension better suited to accommodate the identified use cases were implemented.
The extension meets the following requirements:
\begin{itemize}
    \item The extension targets VS Code 1.32 or newer and is written in Node.js
    \item Automated testing is used wherever possible and meaningful.
    \item The source code is commented and well documented.
\end{itemize}
The thesis was written in parallel with the development of the extension. Git was used as a version control system for the thesis as well as the extension.

The solution was evaluated by comparing the extension and the state-of-the-art tools in terms of suitability for the specified use cases. 
The evaluation is based on the amount of time required to fulfill a specific use case.
Furthermore, any limitations compared to state-of-the-art solutions have been documented.
\chapter{Methodological Approach}

First a state of the art analysis will be conducted. This analysis will look into currently available open-source and commercial products for real time collaboration. 
Additionally the current literature related to real time collaboration will be studied.
The papers mentioned in open source projects will be used as a starting point for forwards and backwards literature analysis.


The following list of publishers will be used as sources of literature: ACM Digital Library\footnote{\href{https://dl.acm.org/}{https://dl.acm.org/}}, IEEE Xplore Digital Library\footnote{\href{https://ieeexplore.ieee.org/}{https://ieeexplore.ieee.org/}}, SpringerLink\footnote{\href{https://link.springer.com/}{https://link.springer.com/}}, and ScienceDirect\footnote{\href{https://www.sciencedirect.com/}{https://www.sciencedirect.com/}}. Google Scholar\footnote{\href{https://scholar.google.com/}{https://scholar.google.com/}} will be used as an additional search engine. Papers that are not published by the previously mentioned publishers will only be used in exceptional cases and after consultation with the supervising assistant.


The literature survey will be continuously documented in the following format:

\begin{lstlisting}
	<date>: <search query>
		<publisher>
			<paper title>
	<search query> := one of the following:
\end{lstlisting}

\begin{itemize}
    \item search terms
    \item conference
    \item forward/backward search paper title
    \item authors of specific papers 
\end{itemize}
If multiple searches occur on the same date, the date has to be included again for every search query.

Based on problems mentioned in papers and issues of projects from the state of the art analysis, requirements for the extension will be defined.
These requirements will be user centered.
Functional as well as non functional aspects such as performance will be defined as requirements.

Use-cases for the extension will also be defined. These use-cases directly related to editing source code will have defined number of users as well as file sizes in order to be verifiable. 
Every use-case includes at least the following information:
\begin{itemize}
    \item Actor (the steakholder)
    \item System (the software the actor is interacting with)
    \item Action (the goal of the actor)
\end{itemize}
A VS Code\footnote{\href{https://code.visualstudio.com/}{https://code.visualstudio.com/}} extension better suited to accomodate the identified use-cases will be implemented.
The extension will meet the following requirements:
\begin{itemize}
    \item The extension will target VS Code 1.32 or newer and will be written in node JS
    \item Automated testing will be used wherever possible and meaningful.
    \item The source code will be commented and well documented.
\end{itemize}
The thesis will be written in parallel with the development of the extension. Git will be used as a version control system for the thesis as well as the extension.

The solution will be evalutated by comparing the extension and the state of the art tools in terms of suitability for the specified use-cases. 
The evaluation will be based on the amount of time required to fulfill a specific use-case.
Furthermore, any limitations compared to state of the art solutions will be documented.